<<<<<<< HEAD
<<<<<<< HEAD
\usepackage[english,francais]{babel} % Pour définir la langue
\selectlanguage{francais}
\newcommand\english[1]{\foreignlanguage{english}{#1}}
\newcommand\français[1]{\foreignlanguage{francais}{#1}}
\usepackage{fontspec}
\setmainfont[
%Ligatures={Common, Rare, TeX, Historical},
%Numbers=OldStyle,
%Contextuals=Swash,
%Style={Historic, Swash}
Ligatures={Common, Rare, TeX},
Numbers=OldStyle,
]
{Junicode}
\usepackage{xeCJK}
\setCJKmainfont[
Style={Historic},
]{教育部標準楷書}

\usepackage{hyperref}
\hypersetup{
	colorlinks=true,       % false: boxed links; true: colored links
    linkcolor=green,          % color of internal links (change box color with linkbordercolor)
    citecolor=magenta,        % color of links to bibliography
    filecolor=red,      % color of file links
    urlcolor=blue
}
\usepackage[stable]{footmisc}

% In-text citation. Il doit sûrement exister un style de citation qui donne le titre et la référence à la première occurence puis seulement la référence.
\newcommand\incite[1]{\english{\citetitle{#1} \cite{#1}}} % in-text citation
=======
=======
>>>>>>> aa25181a1dc2de06adddf3b207744bd4c174dea4
\usepackage[english,francais]{babel} % Pour définir la langue
\usepackage{hyperref,fontspec}
\setmainfont[
%Ligatures={Common, Rare, TeX, Historical},
%Numbers=OldStyle,
%Contextuals=Swash,
%Style={Historic, Swash}
Ligatures={Common, Rare, TeX},
Numbers=OldStyle,
]
{Junicode}
\usepackage{xeCJK}
\setCJKmainfont[
Style={Historic},
]{教育部標準楷書}
\usepackage[stable]{footmisc}
\hypersetup{
	colorlinks=true,       % false: boxed links; true: colored links
    linkcolor=black,          % color of internal links (change box color with linkbordercolor)
    citecolor=black,        % color of links to bibliography
    filecolor=black,      % color of file links
    urlcolor=black
}
<<<<<<< HEAD
\selectlanguage{francais}
>>>>>>> aa25181a1dc2de06adddf3b207744bd4c174dea4
=======
\selectlanguage{francais}
>>>>>>> aa25181a1dc2de06adddf3b207744bd4c174dea4
