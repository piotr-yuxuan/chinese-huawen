% !TeX encoding = UTF-8
\documentclass[12pt,onecolumn]{article} % Type de document
<<<<<<< HEAD
<<<<<<< HEAD
\usepackage[english,francais]{babel} % Pour définir la langue
\selectlanguage{francais}
\newcommand\english[1]{\foreignlanguage{english}{#1}}
\newcommand\français[1]{\foreignlanguage{francais}{#1}}
\input{./preamble.font.tex}

\usepackage{hyperref}
\hypersetup{
	colorlinks=true,       % false: boxed links; true: colored links
    linkcolor=green,          % color of internal links (change box color with linkbordercolor)
    citecolor=magenta,        % color of links to bibliography
    filecolor=red,      % color of file links
    urlcolor=blue
}
\usepackage[stable]{footmisc}

% In-text citation. Il doit sûrement exister un style de citation qui donne le titre et la référence à la première occurence puis seulement la référence.
\newcommand\incite[1]{\english{\citetitle{#1} \cite{#1}}} % in-text citation
=======
=======
>>>>>>> aa25181a1dc2de06adddf3b207744bd4c174dea4
\usepackage[english,francais]{babel} % Pour définir la langue
\usepackage{hyperref,fontspec}
\setmainfont[
%Ligatures={Common, Rare, TeX, Historical},
%Numbers=OldStyle,
%Contextuals=Swash,
%Style={Historic, Swash}
Ligatures={Common, Rare, TeX},
Numbers=OldStyle,
]
{Junicode}
\usepackage{xeCJK}
\setCJKmainfont[
Style={Historic},
]{教育部標準楷書}
\usepackage[stable]{footmisc}
\hypersetup{
	colorlinks=true,       % false: boxed links; true: colored links
    linkcolor=black,          % color of internal links (change box color with linkbordercolor)
    citecolor=black,        % color of links to bibliography
    filecolor=black,      % color of file links
    urlcolor=black
}
<<<<<<< HEAD
\selectlanguage{francais}
>>>>>>> aa25181a1dc2de06adddf3b207744bd4c174dea4
=======
\selectlanguage{francais}
>>>>>>> aa25181a1dc2de06adddf3b207744bd4c174dea4


% About the way bibliography is handled
% Bibliography has been categorised with keywords added according to the following pattern :
% → wanted means I've not been granted read access to the concerned article but I've read already its title and abstract and have found it intereting.
% → read I've processed it already.

\usepackage[
backend=biber,
babel=hyphen,
style=numeric,
sorting=none
]{biblatex}

\addbibresource{../../ref/entries.bib}

%\usepackage{filecontents}
%\begin{filecontents*}{\jobname.bib}
%\input{../../ref/entries.bib}
%\end{filecontents}

\title{Articles recension}
\author{%Pierre de~\textsc{Boisset}~胡雨軒 \href{mailto:p2b.fac@gmail.com}{\texttt{<p2b.fac@gmail.com>}}
%\href{mailto:p2b.fac@gmail.com}{胡雨軒}
胡雨軒\footnote{\href{mailto:p2b.fac@gmail.com}{\texttt{<p2b.fac@gmail.com>}}}}

\begin{document}
\maketitle
\selectlanguage{english}
\begin{abstract}
% This is a \TeX{} disclaimer so I ought dare say I use XeLaTeX.
% My compilation line is very simple: xelatex -synctex=1 -interaction=nonstopmode file.tex

\href{https://creativecommons.org/licenses/by-nc-sa/3.0/}{\textbf{©~CC\textsc{-by-nc-sa}}}\par This has been redacted in English, as all abstracts should be $\sim$ as an effort to keep the general meaning of that document intelligible.\par This is a recension meant to cover each article in the \texttt{ref} directory. First abstract is pasted then I briefly explain why I've chosen to put it in my bibliography. This is aimed at not getting drown in all those deep, difficult yet enthralling articles.\par I'll try for each of them to reply to several questions: what can I use from those for my own aims? What are new suggested projects? Does it make use of ideas I ought use for my own?
\end{abstract}
\selectlanguage{francais}
\tableofcontents

% Take care to keep the section number / citation number binding working.
\section{\citetitle{cook2003unihan}}Il \cite{cook2003unihan}.
\section{\citetitle{li2012chinese}}

\nocite{*}
\printbibliography[title={Bibliographie}, keyword={read}, heading=bibliography] % subbibnumbered
\printbibheading[title={Références à traiter},heading=bibliography]
\printbibliography[title={A télécharger\footnote{Articles et documents auxquels je n'ai pas pu accéder.}}, keyword={wanted}, notkeyword={read}, heading=subbibliography]
\printbibliography[title={A lire\footnote{En d'autres termes : \textsc{taf}}}, notkeyword={wanted}, notkeyword={read}, heading=subbibliography]
\end{document}