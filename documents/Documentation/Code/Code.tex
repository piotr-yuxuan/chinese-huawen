% !TeX encoding = UTF-8
\documentclass[12pt,onecolumn]{article}
\usepackage{import}

\newcommand\directory{../../}
\import{\directory/tex/preamble/}{general.tex}
\import\preamble{dev.tex}

\title{Rendu de \textsc{inf723}}
\author{Pierre de \textsc{Boisset}}

\begin{document}
\selectlanguage{francais}

\maketitle

L'intégralité des sources de ce projet est disponible dans le dépôt \href{https://github.com/piotr2b/chinese-huawen}{github associé}. Un guide rapide d'installation y est également disponible.

Le cahier des charges de ce projet y est également disponible dans le fichier \href{https://github.com/piotr2b/chinese-huawen/raw/master/documents/Introduction/Introduction.pdf}{Introduction.pdf}.

Le projet Java y est situé dans le dossier \href{https://github.com/piotr2b/chinese-huawen/tree/master/java/parser}{/java/parser}. Ce répertoire pesant 83 Mio et le dépôt github au total plus de 300 Mio, j'ai préféré ne pas le mettre sur le site \url{http://services.infres.enst.fr/rendutp/}.

Ce document de fin de projet en montre différentes facettes. Après une présentation générale du contexte plus détaillée que ce qui avait été rédigée pour le cahier des charges, nous expliquerons certains choix techniques et décrirons l'agencement des classes qui composent ce projet. Nous terminerons par des \textsc{scenarii} d'utilisation possibles.

Le site \url{piotr.github.io} présente le contexte dans lequel s'inscrit ce projet, la structure des caractère chinois et les différents types d'arbre qu'on peut construire pour un caractère. Cette introduction détaillée fait partie du présent document qu'elle est censée éclairer.

\end{document}