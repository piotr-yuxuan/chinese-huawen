\section{\LaTeX{}}

\subsection{Why using \LaTeX{} instead of markdown?}

You're right, markdown can be read directly on github with no downloads and is lighter than \LaTeX{} but the latter is pretty easier than markdown in purpose of typesetting hypertext including links, bibliographies and so on. Documents written with \LaTeX{} are far more beautiful and so much easier to compose instead of being a few bit slower to get typed.

Nonetheless I've been trying to automatically generate markdown files with pandoc then to not to upload `pdf` here but just \texttt{*.tex} and \texttt{*.md}. I faced some issues (handling bibliographiy-related and \texttt{\textbackslash{}input} commands) then I gave up. I know you more than me about it, please tell me :-)

\subsection{Preamble}

\LaTeX{} files here use preambles as modules, so rather prefer to load preambles with commands from the package \href{http://ctan.mines-albi.fr/macros/latex/contrib/import/import.pdf}{\texttt{import}}. On the contrary, it's easier to enclose text with \texttt{\textbackslash{}input} or even \texttt{\textbackslash{}inputAllFiles} which import \texttt{*.tex} files from a directory and sort them lexicographically.

\subsection{Babel and \textsc{cjk}\footnote{Stands for \textsl{C}hinese, \textsl{J}aponese and \textsl{K}orean.}}

A sinogram in standard text is rendered correctly but when typed in \texttt{\textbackslash{}foreignlanguage} just produces a \français{華} character. Hum\dots{}, it seems it was a bug and it's been fixed.

\subsection{Font and \textsc{idc}\footnote{Stands for \textsl{i}deographic \textsl{d}escription \textsl{c}haracter.}}

Using \texttt{xe\textsc{cjk}} is bad and short-minded:
\begin{lstlisting}[caption={}]
\usepackage{fontspec,xeCJK}

\setmainfont{Linux Libertine}
\setCJKmainfont{AR PL New Sung}
\end{lstlisting}

It's better to use \texttt{ucharclasses}. Two characters from different unicode block may be separated by a blank or enclosed in \{.\} because that package doesn't change font inside a word. I think it's a bug. Refer to \texttt{mw⿲ce.tex} to see how to use it.

It has been put in general \texttt{font.tex} preamble.

Methinks the compiler automatically embeds needed font glyphs so no need to add arguments like \texttt{-dSubsetFonts=true} \texttt{-dEmbedAllFonts=true}.

\subsection{Compilation}

I use the Xe\TeX{} compiler.

The option \texttt{recorder} is required by the package \href{http://www.ctan.org/tex-archive/macros/latex/contrib/currfile}{\texttt{currfile}}. More specifically, the \texttt{abspath} option loads the sub-package \texttt{currfile-abspath} and requires compiler option \texttt{recorder} to be used. Thus, as \texttt{currfile} \textsl{could be used}, your compilation line should look comething like:

\begin{lstlisting}[caption={}]
xelatex -recorder
\end{lstlisting}
